%-------------------------------------------------------------------------------------------------
% Template Version 1.0.4							27.02.2015
% 1.0.1		einfache Erg�nzungen und Korrekturen (begin{Citation}, etc.), FOM Anpassungen am Titelblatt
% 1.0.2		\Zitat{}
% 1.0.3		minipage im Environment graybox
% 1.0.4		Schriftarten eingef�gt
% Achtung! in der Datei dinat.bst wurde die Fkt. push.ua von "u.\,a." in "et~al." ge�ndert ! - kein Versionsinkrement n�tig - 
% 1.0.5		\dcite f�r direkte Zitate ohne "Vgl."
% 1.0.6		R�mische Verzeichnisnummerierung, Aufbau der Arbeit, Erg�nzung in Kapiteln, Verzeichnisbezeichnungen korrigiert, Tiefe ToC
%-------------------------------------------------------------------------------------------------

% Optionen: coloredlinks, noheadrule, nochapter, moderntitle, classictitle, lockflag, affidavit
\documentclass[classictitle,affidavit]{fomsdt}

% WENN Inhaltsverzeichnis, Abk�rzungsverzeichnis, Abbildungsverzeichnis, Tabellenverzeichnis AM Dokumentenanfang stehen sollen DANN
 \usepackage[abbreviation]{fomdss} % optionen: toctotoc (TOC ins TOC aufnehmen), nopagebreak (Seitenumbruch nach Verzeichnissen), abbreviation (Abk�rzungsverzeichnis), abbrevrefpage(Ref zur Seite)
%-------------------------------------------------------------------------------------------------
% WENN Inhaltsverzeichnis, Abk�rzungsverzeichnis, Abbildungsverzeichnis, Tabellenverzeichnis AM Dokumentenende stehen sollen DANN
 \usepackage[glossary,index]{fomdes} % optionen: listings (Codelisting etc), glossary, glossrefpage, nobibliography, index, nopagebreak
%-------------------------------------------------------------------------------------------------
% WENN Abs�tze, Zitate (besondere) DANN
 \usepackage{fompar} % keine optionen
%-------------------------------------------------------------------------------------------------
% WENN Abbildungen, Tabellen, Codelistings, Graphics DANN
 \usepackage{fomfod}
%-------------------------------------------------------------------------------------------------
% WENN NormDatum, spez. Symbole (z.B. Copyright etc) DANN
% \usepackage{fomfod}
%-------------------------------------------------------------------------------------------------
% Das Paket fomidx steckt in der Dokumentenklasse (Ausnahme!) da es den Regeln von LaTeX geschuldet ist, geh�rt aber zum Paket Dokumentenende
%-------------------------------------------------------------------------------------------------
\usepackage{xstring} % f�r macro \fcite bzw. \dcite um letztes Zeichen zu extrahieren 
%-------------------------------------------------------------------------------------------------
% SCHRIFTARTEN------------------------------------------------------------------------------------
%\usepackage{bookman}
 %\usepackage{lmodern}
 %\usepackage{mathpazo}
 %\usepackage{kpfonts}
 %\usepackage{mathptmx}
 %\usepackage{times,mtpro2}
 %\usepackage{stix}
 %\usepackage{txfonts}
 %\usepackage{newtxtext,newtxmath}
 %\usepackage{libertine} \usepackage[libertine]{newtxmath}
%-------------------------------------------------------------------------------------------------
%-------------------------------------------------------------------------------------------------
% INHALTSVERZEICHNIS EIGENSCHAFTEN ---------------------------------------------------------------
%\setcounter{tocdepth}{5} % damit \paragraph und \subparagraph in das Inhaltsverzeichnis mit aufgenommen werden
%\setcounter{secnumdepth}{5} % damit \paragraph und \subparagraph als 4. und 5. Ebene gelten
%-------------------------------------------------------------------------------------------------


\begin{document}

%-TITELSEITE--------------------------------------------------------------------------------------
%-------------------------------------------------------------------------------------------------
	% WENN documentclass:classictitle
	\Institute{Hochschule f�r\\ �konomie \& Management Essen} 	% Name der Hochschule
	\Course{Berufsbegleitender Studiengang zum Bachelor of Science}			% optional : z.B. berufsbegleitender Studiengang
	\Semester{X. Semester}							% 
	\Subject{Seminararbeit in Modul}							% z.B. Diplomarbeit zum Thema
	\Title{Titel: Subtitel}	% Thema der Arbeit
	\Professor{Dozent*in}						% Dozent
	\Student{Erhan Tuna}		
	%\Matrikel{ab321194}
	\Date{\today}										% pflicht: standard = \today
	%-------------------------------------------------------------------------------------------------
	% + WENN documentclass:moderntitle
	\General{Hausarbeit} % Diplomarbeit, Studienarbeit, Hausarbeit, etc..
	%-------------------------------------------------------------------------------------------------
	% WENN documentclass:lockflag
	%\LockflagContent[]{fom} % fom steht f�r Standard-FOM-Formulierung; sig f�r Ort, Datum, Unterschrift
	%-------------------------------------------------------------------------------------------------
	% WENN documentclass:affidavit
	\AffidavitContent[]{std} % std ist die Standard-Formulierung
	%-------------------------------------------------------------------------------------------------
	% WENN compiler=pdflatex DANN (optional)
	\PDFTitle{Titel: Subtitel}
	\PDFAuthor{Erhan Tuna}
	\PDFSubject{...arbeit}
	\PDFProducer{MikTeX}
	%-------------------------------------------------------------------------------------------------
	% Erstellt Titelseite, Sperrvermerk, Eid. Versicherung
	\maketitle
%---------------------------------------------------------------------------------------------------
%---------------------------------------------------------------------------------------------------

%-----------------------------------------------------------
% Beginne mit den Verzeichnissen
%-----------------------------------------------------------
\pagenumbering{Roman} % Roman = I,II,.. und roman = i,ii,... 

%---------------------------------------------------------------------------------------------------
%-VERZEICHNISSE
%---------------------------------------------------------------------------------------------------
\PrintContents[Inhaltsverzeichnis] % TOC
\PrintAbbrev[Abk�rzungsverzeichnis]	% Abk�rzungsverzeichnis
\PrintFigures[Abbildungsverzeichnis] % Abbildungsverzeichnis
\PrintTables[Tabellenverzeichnis]	% Tabellenverzeichnis
%---------------------------------------------------------------------------------------------------


%---------------------------------------------------------------------------------------------------
%-Seitennummerierung
%---------------------------------------------------------------------------------------------------
\pagenumbering{arabic} % Arabisch 1,2,... (Beginned bei 1)
\onehalfspacing       % Ab hier Zeilenabstand �ndern (1.5-zeilig)
%\setstretch{1.2}      % Ab hier Zeilenabstand �ndern (1.2-zeilig)
%---------------------------------------------------------------------------------------------------


%---------------------------------------------------------------------------------------------------
%-EIGENE MAKROS
%---------------------------------------------------------------------------------------------------
% #1: Bookname, #2: Seitenangabe
\newcommand{\dcite}[2]{\footnote{\raggedright{\citeauthor{#1}~(\citeyear{#1}),~S.~#2.}}} % bei direkten Zitaten %\ifthenelse{\equal{\StrRight{#2}{1}}{.}}{}{.}
\newcommand{\fcite}[2]{\footnote{\raggedright{Vgl. \citeauthor{#1}~(\citeyear{#1}),~S.~#2.}}} % bei indirekten Zitaten
%----------------------------------------------------------------------------------------------------
% zentrierte Zitate vereinfachen
% Definieren von else then
%\def\ifemptyarg#1{%
  %\if\relax\detokenize{#1}\relax % H. Oberdiek
    %\expandafter\@firstoftwo
  %\else
    %\expandafter\@secondoftwo
  %\fi}

%\def\Zitat#1
%{	
	%\begin{Citation}[\parskip]{12mm}{15mm}
	%\textit{\glqq #1\grqq}
	%\end{Citation}
%}
%\ifemptyarg{#2}{}{\flushright{#2}}
%\newcommand{\Zitat}{1} % Diese Version geht aus Compilertechnischen Gr�nden nicht, siehe ftp://ftp.ams.org/pub/tex/amslatex/math/technote.pdf (Punkt 6) und http://www.golatex.de/wiki/%5Cdef
%{
	%\begin{Citation}[\parskip]{12mm}{15mm}
	%\textit{\glqq #1 \grqq}
	%\end{Citation}
%}
\newcommand{\Zitat}[2] % keine Verwendung von fcite in 2. Argument m�glich, kein Anh�ngen von fcite an das Zitat m�glich (neue zeile)
{						% Einzige Besonderheit von \Zitat ist der Urheber/Autor rechts vom Zitat
  \begin{list}{}
   {
		\setlength{\partopsep}{\parskip}%
		\setlength{\topsep}{0pt}%
		\setlength{\parsep}{\parskip}%
		\setlength{\itemsep}{0pt}%
		\setlength{\listparindent}{0pt}%
		\setlength{\labelwidth}{0pt}%
		\setlength{\labelsep}{0pt}%
		\setlength{\leftmargin}{12mm}%
		\setlength{\rightmargin}{15mm}%
	}
	\item \textit{\glqq #1\grqq}
	\ifthenelse{\equal{#2}{}}{}{\flushright{\tiny- #2 -}} % {condition}{true}{false}
  \end{list}
}

%----------------------------------------------------------------------------------------------------
% Pfade zu Graphiken setzten.
%----------------------------------------------------------------------------------------------------
\graphicspath{{./img/}}  % Pfade bekannt geben (ACHTUNG: Syntax beachten)
%----------------------------------------------------------------------------------------------------
% Literaturverzeichnis et al. statt u.a.
%\DefineBibliographyStrings{ngerman}{andothers={et\addabbrvspace al\adddot}}      % et al. statt u.a.-> funzt nicht mit dinat
%----------------------------------------------------------------------------------------------------
% F�r Hervorhebungen, Zitate und K�sten(?) mit Text, siehe folgende Makros
%\Remark[Einzug][Abstand]{\bf label}{text}	% linksb�ndiges label
%\Item[Einzug][Abstand]{\bf label}{text}	% rechtsb�ndiges label
%\begin{Citation}[soll optional, ist aber pflicht]{pflicht}{pflicht} % Abstand links und rechts; Case-sensitive
%\end{Citation}
% fompar:\Description f�r �berschriften, die nicht ins TOC sollen (keine Verwendung von sub/paragraph)
%----------------------------------------------------------------------------------------------------

%----------------------------------------------------------------------------------------------------
% Box (minipage) mit farbigem Hintergund (RFU)
%----------------------------------------------------------------------------------------------------
\makeatletter
\newlength{\graybox@border}
%\definecolor{graybox@color}{RGB}{241,241,241}% light gray backgroud
\definecolor{graybox@color}{RGB}{232,232,232}% light gray backgroud
\newenvironment{graybox}[1][0pt]{%
  \setlength{\graybox@border}{#1}%            save border size
  \begin{lrbox}{\@tempboxa}%                  begin "lrbox"
    \begin{minipage}{\linewidth}%             begin outer "minipage"
    \addtolength{\linewidth}{-2\graybox@border}% left/right border
    \vspace{\graybox@border}\centering%       leading border
    \begin{minipage}{\linewidth}%             begin inner "minipage"
}{%
  \end{minipage}%                             end inner "minipage"
  \vspace{\graybox@border}%                   following border
  \end{minipage}%                             end outer "minipage"
  \end{lrbox}%                                end "lrbox"
  \colorbox{graybox@color}{\usebox{\@tempboxa}}% print box data!
}\makeatother%
%----------------------------------------------------------------------------------------------------
%----------------------------------------------------------------------------------------------------


%----------------------------------------------------------------------------------------------------
%----------------------------------------------------------------------------------------------------
%-----------------------------------------------------------
% 
% Datei:        			Einleitung.tex
% 
% Beschreibung: 			Kapitel: Einleitung
% 
% Autor:        			Erhan Tuna
% 
% Version Einleitung:      	1.0.4
%
% Anforderungen:
% Argumentationskette schl�ssig
% Beispiele vorhanden
% Prim�rliteratur ist vorzuziehen
% Roter Faden
%-----------------------------------------------------------

%-Leitfaden wissenschaftliche Arbeit FOM--------------------
% Wenn Prim�rliteratur nicht abrufbar, dann \footnote{Originalquelle zitiert nach Sekundarquelle}
% Wenn mehrfach auf einer Seite die selbe Quelle genutzt wird, ist es mit ebd. abzuk�rzen
% Englische Zitate k�nnen als Original aufgef�hrt werden, alle anderen Sprachen sind zu �bersetzen	
% Internetquellen m�ssen als PDF-Export mit abgegeben werden (digital)
%-----------------------------------------------------------

%-Demoabsatz f�r Funktionen-------------------------------------------------------------------------
%Dies ist ein Zitat: \fcite{default}{12-15} \\
%Die Erkl�rung des Wortes \gloss[short]{std}. \\
%Eine Abk�rzung w�re die \Abbrev{PIN}{\Mark{P}ersonal \Mark{I}dentification \Mark{N}umber}\index{PIN}. \\
%Eine weitere Quellenangabe sieht so aus.\fcite{Jager}{12 ff} \\
%Index\index{Index} ist ein wichtiges Wort.
%Wichtige Worte werden in das Indexverzeichnis\index{Index!verzeichnis} aufgenommen. \\
%\newpage
%Werden wichtige Worte, wie Indexverzeichnis\index{Index!verzeichnis}, auch ein 2. Mal automatisch aufgenommen? nein\\
%W�rter die mit Index beginnen, Indexschulung\index{Index!schulung}.
%Wird eine Quelle mehrfach hintereinander genannt, so kann das mit "ebd." abgek�rzt werden. "f." "ff." mit Punkt
%---------------------------------------------------------------------------------------------------

\section{Einleitung}
% definieren, erkl�ren, abgrenzen

Dies ist ein Zitat.\fcite{default}{12-15} \\
Die Erkl�rung des Wortes \gloss[short]{std}. \\
Eine Abk�rzung w�re die \Abbrev{PIN}{\Mark{P}ersonal \Mark{I}dentification \Mark{N}umber}\index{PIN}. \\
Eine weitere Quellenangabe sieht so aus.\fcite{Jager}{12 ff.} \\
Index\index{Index} ist ein wichtiges Wort.
Wichtige Worte werden in das Indexverzeichnis\index{Index!verzeichnis} aufgenommen. \\
\newpage
Werden wichtige Worte, wie Indexverzeichnis\index{Index!verzeichnis}, auch ein 2. Mal automatisch aufgenommen? \\
W�rter die mit Index beginnen, Indexschulung\index{Index!schulung}.
\SinglePicture[0.7][box]{Das Wasserfallmodell}{fig:water}{Wasserfall.png}{Anhand o.g. Entwicklungsphasen}

%-----------------------------------------------------------
% EOF
%-----------------------------------------------------------


%----------------------------------------------------------------------------------------------------
%----------------------------------------------------------------------------------------------------
%-----------------------------------------------------------
% 
% Datei:        Hauptteil.tex
% 
% Beschreibung: Kapitel: Hauptteil
% 
% Autor:        Erhan Tuna
% 
% Version:      1.0.0
%
%-----------------------------------------------------------


\section{Hauptteil}


%----------------------------------------------------------------------------------------------------
%----------------------------------------------------------------------------------------------------
%-----------------------------------------------------------
% 
% Datei:        Fazit.tex
% 
% Beschreibung: Kapitel: Fazit
% 
% Autor:        Erhan Tuna
% 
% Version:      1.0.1 Aufbau der Arbeit erklärt
% Version:      1.0.0
%
% Hinweise:
% Gar nichts neues mehr; Zukunftsausblick; Resumeé aus Kapitel 3
%-----------------------------------------------------------

%-----------------------------------------------------------
% Aufbau der Arbeit
% 1. Einleitung (1 Seite): ganz allgemein auf das Thema eingehen \par Problemstellung \par Aufbau der Arbeit 
% 2. Begriffe erklären: z.B. Beschaffung, Management, dann Beschaffungsmanagement
% 3. kritische Betrachtung: das 3. Kapitel ist immer das wichtigste Kapitel
% 4. Fazit/Ausblick: am Besten auf vorher genannte Argumente eingehen
%-----------------------------------------------------------

\section{Fazit/Ausblick}


%-----------------------------------------------------------
% EOF
%-----------------------------------------------------------

%----------------------------------------------------------------------------------------------------
%----------------------------------------------------------------------------------------------------

%---------------------------------------------------------------------------------------------------
%-DOKUMENT ENDE (fomdes)
%---------------------------------------------------------------------------------------------------
%\PrintListings % Codelistings
%\PrintGlossary{Glossar}	% Glossar{Dateiname ohne Endung} % optionen: all
\PrintBibliography[Literaturverzeichnis]{Literatur} % Literaturverzeichnis{Dateiname ohne Endung}
%\PrintIndex[Index][double]		% Stichwortverzeichnis % optionen: single, double, triple (=Anzahl der Spalten)
%---------------------------------------------------------------------------------------------------

\end{document}