%-----------------------------------------------------------
% 
% Datei:        			Einleitung.tex
% 
% Beschreibung: 			Kapitel: Einleitung
% 
% Autor:        			Erhan Tuna
% 
% Version Einleitung:      	1.0.4
%
% Anforderungen:
% Argumentationskette schl�ssig
% Beispiele vorhanden
% Prim�rliteratur ist vorzuziehen
% Roter Faden
%-----------------------------------------------------------

%-Leitfaden wissenschaftliche Arbeit FOM--------------------
% Wenn Prim�rliteratur nicht abrufbar, dann \footnote{Originalquelle zitiert nach Sekundarquelle}
% Wenn mehrfach auf einer Seite die selbe Quelle genutzt wird, ist es mit ebd. abzuk�rzen
% Englische Zitate k�nnen als Original aufgef�hrt werden, alle anderen Sprachen sind zu �bersetzen	
% Internetquellen m�ssen als PDF-Export mit abgegeben werden (digital)
%-----------------------------------------------------------

%-Demoabsatz f�r Funktionen-------------------------------------------------------------------------
%Dies ist ein Zitat: \fcite{default}{12-15} \\
%Die Erkl�rung des Wortes \gloss[short]{std}. \\
%Eine Abk�rzung w�re die \Abbrev{PIN}{\Mark{P}ersonal \Mark{I}dentification \Mark{N}umber}\index{PIN}. \\
%Eine weitere Quellenangabe sieht so aus.\fcite{Jager}{12 ff} \\
%Index\index{Index} ist ein wichtiges Wort.
%Wichtige Worte werden in das Indexverzeichnis\index{Index!verzeichnis} aufgenommen. \\
%\newpage
%Werden wichtige Worte, wie Indexverzeichnis\index{Index!verzeichnis}, auch ein 2. Mal automatisch aufgenommen? nein\\
%W�rter die mit Index beginnen, Indexschulung\index{Index!schulung}.
%Wird eine Quelle mehrfach hintereinander genannt, so kann das mit "ebd." abgek�rzt werden. "f." "ff." mit Punkt
%---------------------------------------------------------------------------------------------------

\section{Einleitung}
% definieren, erkl�ren, abgrenzen

Dies ist ein Zitat.\fcite{default}{12-15} \\
Die Erkl�rung des Wortes \gloss[short]{std}. \\
Eine Abk�rzung w�re die \Abbrev{PIN}{\Mark{P}ersonal \Mark{I}dentification \Mark{N}umber}\index{PIN}. \\
Eine weitere Quellenangabe sieht so aus.\fcite{Jager}{12 ff.} \\
Index\index{Index} ist ein wichtiges Wort.
Wichtige Worte werden in das Indexverzeichnis\index{Index!verzeichnis} aufgenommen. \\
\newpage
Werden wichtige Worte, wie Indexverzeichnis\index{Index!verzeichnis}, auch ein 2. Mal automatisch aufgenommen? \\
W�rter die mit Index beginnen, Indexschulung\index{Index!schulung}.
\SinglePicture[0.7][box]{Das Wasserfallmodell}{fig:water}{Wasserfall.png}{Anhand o.g. Entwicklungsphasen}

%-----------------------------------------------------------
% EOF
%-----------------------------------------------------------

