%---------------------------------------------------------------------------------------------------
% 
% Datei:        			Einleitung.tex
% 
% Beschreibung: 			Kapitel: Einleitung
% 
% Autor:        			Erhan Tuna
% 
% Version Einleitung:      	1.1.0	- Demoabsatz angepasst, Rezept erweitert, Roter Faden erkl�rt, 
%																		Gliederungsaufbau allgemein erzeugt
% Version Einleitung:      	1.0.4
%
% Anforderungen:
% Argumentationskette schl�ssig
% Beispiele vorhanden
% Prim�rliteratur ist vorzuziehen
% Roter Faden (heisst, man muss von einem Thema zum anderen schlie�en k�nnen, wenn man die Gliederung liest)
%------------------------------------------------------------

% Roter Faden------------------------------------------------
% richtet sich nach der Schl�ssigkeit und den Aufbau der Gliederung
% z.B. Begriffe erkl�ren, dann benutzen
% lt. Prof. Dr. Nguyen zielt das nicht auf den Inhalt des Textes ab, es geht um die Gliederung
%------------------------------------------------------------

%------------------------------------------------------------
% Aufbau der Arbeit
% 1. Einleitung (1 Seite): ganz allgemein auf das Thema eingehen \par Problemstellung \par Aufbau der Arbeit 
% 2. Begriffe erkl�ren: z.B. Beschaffung, Management, dann Beschaffungsmanagement
% 3. kritische Betrachtung: das 3. Kapitel ist immer das wichtigste Kapitel
% 4. Fazit/Ausblick: am Besten auf vorher genannte Argumente eingehen
%------------------------------------------------------------
% 1. Einleitung (1 Seite): 	1. Absatz: ganz allgemein zum Thema reden 
%														2. Absatz: Problemstellung (Neugier wecken) 
%														3. Absatz: Vorgehensweise in Arbeit
% 2. Begriffe erkl�ren: z.B. 	1. Absatz: Beschaffung 
%															2. Absatz: Management 
%															3. Absatz: Beschaffungsmanagement
% 3. 	Kritische Betrachtung (wichtigstes Kapitel): Eingehen auf die Probleme, vergleichen, 
%			Pro- und Kontra
%			Themen klassifizieren
%------------------------------------------------------------

%-REZEPT zur VORGEHENSWEISE BEIM ERSTELLEN DER ARBEIT--------
% 1. In das Thema einlesen
% 2. Gliederung aufbauen
% 3. wichtige Schl�sselw�rter und Argumente passen zum Thema und mit Quellen notieren
% 4. Mit kritischer Betrachtung beginnen und Abs�tze schreiben, dabei auf Schl�ssigkeit 
%		achten und sie stand-alone-plausibel machen
% 	auf wenige Behauptungen konzentrieren; die daf�r detailiert er�rtern;
%		Argumentationskette beachten 
%-----------------------------------------------------------

% Auf wenige Behauptungen konzentrieren und die daf�r detailiert er�rtern
% Mit der krisitschen Betrachtung beginnen, Fragen von ausgesuchten Behauptungen beantworten,
% danach die Einleitung schreiben
% Nicht so viel im klein klein verheddern, sondern nach dem Einlesen sofort die Themen herauspicken
% und sich ausschlie�lich darauf konzentrieren - das spart Zeit
% Am Ende f�llt man die Seiten mit etwas l�ngeren Formulierungen
% Von Anfang an sauber zitieren und schreiben, damit keine weiteren iterativen Korrekturen n�tig sind
%-----------------------------------------------------------

%-Leitfaden wissenschaftliche Arbeit FOM--------------------
% Wenn Prim�rliteratur nicht abrufbar, dann \footnote{Originalquelle zitiert nach Sekundarquelle}
% Wenn mehrfach auf einer Seite die selbe Quelle genutzt wird, ist es mit ebd. abzuk�rzen
% Englische Zitate k�nnen als Original aufgef�hrt werden, alle anderen Sprachen sind zu �bersetzen	
% Internetquellen m�ssen als PDF-Export mit abgegeben werden (digital)
%-----------------------------------------------------------

%-Demoabsatz f�r Funktionen---------------------------------
%Dies ist ein Zitat: \fcite{<quellname>}{<seite>} \\
%Dies ist ein w�rtliches Zitat: \dcite{<quellname>}{<seite>}
%Eine weitere M�glichkeit ist:\footnote{Vgl. <NameVerfasser> (Jahr), S. 9 ff.} \\
%Glossareintrag \gloss[short]{std}. \\
%Eintrag in Abk�rzungsverzeichnis \Abbrev{PIN}{\Mark{P}ersonal \Mark{I}dentification \Mark{N}umber}\index{PIN}. \\
%Index\index{Index} ist ein wichtiges Wort.
%Wichtige Worte werden in das Indexverzeichnis\index{Index!verzeichnis} aufgenommen. \\
%\newpage
%Werden wichtige Worte, wie Indexverzeichnis\index{Index!verzeichnis}, auch ein 2. Mal automatisch aufgenommen? nein\\
%W�rter die mit Index beginnen, Indexschulung\index{Index!schulung}.
%Wird eine Quelle mehrfach hintereinander genannt, so kann das mit "ebd." abgek�rzt werden. "f." "ff." mit Punkt
%\Description[<stil>]{<title>}{<text>}
%------------------------------------------------------------

\section{Einleitung}
%----------------------------------------------------------------------
% AUFBAU EINLEITUNG
%----------------------------------------------------------------------
% 3 Abs�tze mit den zuvor definierten Unterpunkten, allerdings keine Unterpunkte
% Einleitung, Problemstellung, Motivation/Zielsetzung
%----------------------------------------------------------------------
% Ziel der Einleitung
%----------------------------------------------------------------------
% Hinf�hrung zum Thema und INTERESSE des Lesers wecken, SPANNUNG aufbauen (vgl. Emotionale Rhetork; Herrmann-Ruess, Anita (2014), S. 158)
% Aufbau und Schreibstil der Einleitung:
% "Folgt man dem Wissenschaftsphilosophen Karl Popper, so beginnt jede Erkenntnis mit der Wahrnehmung eines Problems. 
% Also bietet es sich an, diesen Ansto� zur Durchf�hrung einer wissenschaftlichen Arbeit einleitend zu erl�utern und zu begr�nden, 
% WARUM die L�sung des Problems wichtig ist � mehr nicht. So wie der Autor eines Krimis in der Regel nicht schon in der ersten Szene verr�t, 
% wer der M�rder ist, sollte der Autor einer wissenschaftlichen Abschlussarbeit nicht schon in der Einleitung alles verraten. Auch eine 
% gute Abschlussarbeit zeichnet sich dadurch aus, dass sie zu Beginn Spannung aufbaut und diese bis zum Schluss aufrechterh�lt." 
% (vgl. https://www.audimax.de/studium/wissenschaftliches-arbeiten/tipps/perfekte-gliederung-in-sieben-phasen/)
%----------------------------------------------------------------------
% Beispiele:
%----------------------------------------------------------------------
% ...aber wie sieht es mit den unterschiedlichen Qualit�tsstandards und Normen der L�nder aus?
% Aber lohnt es sich wirklich, in den L�ndern einzukaufen, die die besten Preise bieten?
%----------------------------------------------------------------------
% Bei einer kleinen Seminararbeit und nur 10-12 Seiten Platz, 
% macht man keine Unterpunkte zur Einleitung, sondern geht her und schreibt
% 3 Abs�tze mit jeweils den Themen: allgemeine Einleitung zum Thema \par Problemstellung \par Aufbau der Arbeit
% Insgesamt gilt auch, keine W�rter benutzen, die man nicht erkl�rt
%----------------------------------------------------------------------




%-----------------------------------------------------------
% EOF
%-----------------------------------------------------------

