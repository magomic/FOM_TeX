%-----------------------------------------------------------
% 
% Datei:        KritischeBetrachtung.tex
% 
% Beschreibung: Kapitel 3: Kritische Betrachtung
% 
% Autor:        Erhan Tuna
% 
% Version:      1.1.0	Hinweise eingefügt
% Version:      1.0.1	Aufbau erklärt
%
% Hinweise:
% Auf wenige Behauptungen konzentrieren und die dafür detailiert erörtern
% Mit der krisitschen Betrachtung beginnen, Fragen von ausgesuchten Behauptungen beantworten,
% danach die Einleitung schreiben
% Nicht so viel im klein klein verheddern, sondern nach dem Einlesen sofort die Themen herauspicken
% und sich ausschließlich darauf konzentrieren - das spart Zeit
% Am Ende füllt man die Seiten mit etwas längeren Formulierungen
% Von Anfang an sauber zitieren und schreiben, damit keine weiteren iterativen Korrekturen nötig sind
%-----------------------------------------------------------


%-----------------------------------------------------------
% Aufbau der Arbeit
% 1. Einleitung (1 Seite): ganz allgemein auf das Thema eingehen \par Problemstellung \par Aufbau der Arbeit 
% 2. Begriffe erklären: z.B. Beschaffung, Management, dann Beschaffungsmanagement
% 3. kritische Betrachtung: das 3. Kapitel ist immer das wichtigste Kapitel
% 4. Fazit/Ausblick: am Besten auf vorher genannte Argumente eingehen
%-----------------------------------------------------------


\section{Hauptteil}

